%%=============================================================================
%% Conclusie
%%=============================================================================

\chapter{Conclusie}%
\label{ch:conclusie}

Deze bachelorproef had als doel een methode te ontwikkelen om de observatievaardigheden van 
studenten in het 360° Zorglab van HOGENT op een geautomatiseerde, objectieve manier te evalueren.
Computervisie-modellen werden geïntegreerd met eyetrackingdata van Tobii Glasses, om zo de huidige, 
subjectieve evaluatiemethode te vervangen door een datagestuurde aanpak.
Hiertoe werd een proef-of-concept applicatie ontwikkeld en werden verschillende computervisie-modellen 
geëvalueerd op hun vermogen om objecten te detecteren en te segmenteren in de eyetrackingdata.

\section{Beantwoording van de Onderzoeksvragen}

De centrale onderzoeksvraag was: 
\textit{Hoe kunnen computervisie-modellen geïntegreerd worden met eyetrackingdata van Tobii Glasses om observatieprestaties 
van studenten in het 360° Zorglab automatisch te analyseren?}
Er werd aangetoond dat een dergelijke integratie haalbaar is en bovendien veelbelovende resultaten oplevert.
De deelvragen kunnen als volgt beantwoord worden:

\paragraph{\textit{Welke barrières (cognitief, technisch of didactisch) ervaren trainers en studenten bij de huidige, handmatige observatiemethodes?}}
Deze vraag werd binnen deze proef niet expliciet onderzocht via een nieuw, formeel gebruikersonderzoek. 
Desondanks konden de inherente barrières worden geïdentificeerd. 
Dit gebeurde op basis van een combinatie van de initiële probleemstelling, een grondige literatuurstudie (Sectie~\ref{sec:eyetracking}) en inzichten verkregen uit overleg met de co-promotor. 
Deze laatste is zelf een ervaren docent in de verpleegkunde en is de opdrachtgever van dit onderzoek. 
De voornaamste barrières die hieruit naar voor kwamen, waren de subjectiviteit in de beoordeling, 
de tijdrovende aard van zelfrapportage en directe, observatie evenals het gebrek aan objectieve data over kijkgedrag.  
    
\paragraph{\textit{Welke kenmerken moet een geautomatiseerde analysemethode hebben om de huidige beperkingen van handmatige observatie te verhelpen?}}
In een ideale wereld zou een volledig geautomatiseerde analysemethode in staat zijn  
de eyetrackingopnames van studenten volledig te analyseren zonder enige menselijke tussenkomst.
Echter, gezien de huidige stand van de technologie, is dit nog niet haalbaar.
Daarom werd er gekozen voor een semi-automatische aanpak, waarbij de trainer nog steeds een actieve rol speelt in het initialiseren van de analyse.
In de ontwikkelde pipeline voert de trainer de `kalibratiestap' uit, waarbij de kritische objecten worden gelabeld. 
De ontwikkelde methode omvat daarom als ondersteunend kenmerk, een softwareapplicatie (Hoofdstuk~\ref{ch:ontwikkeling}) met een gebruiksvriendelijke interface. 
Deze stelt trainers in staat eyetrackingopnames te beheren, objecten te definiëren en deze efficiënt te labelen.
Aangezien computer-vision een veld is dat voortdurend evolueert, was het ook belangrijk dat de ontwikkelde software voortdurend kon bijgestuurd worden.
De software is daarom modulair opgebouwd (volgens Sectie~\ref{sec:software-architectuur}), zodat nieuwe componenten gemakkelijk kunnen worden toegevoegd.

\paragraph{\textit{In welke mate kunnen de modellen en de ontwikkelde software:}}
\begin{enumerate}
    \item \textit{correct bepalen welke kritische objecten studenten hebben waargenomen?}
    \item \textit{nauwkeurig meten hoe lang studenten naar deze objecten keken?}
\end{enumerate}
De beste resultaten werden behaald door de combinatie van een YOLOv11-object\-detector getraind op 1000 samples per klasse, gecombineerd met FastSAM-tra\-cking (zie Sectie~\ref{sec:analyse-beste-model}).
Deze combinatie behaalde een micro-gemiddelde F1-score van 0.8017, met een precisie van 0.9424 en een recall van 0.6976.
Dit duidt erop dat de software met hoge precisie objecten kan identificeren (weinig fout-positieven), 
maar dat er nog ruimte voor verbetering is in de recall (het detecteren van alle daadwerkelijk bekeken objecten). 
De analyse van vals-positieven toonde aan dat een aanzienlijk deel hiervan correct gedetecteerde objecten waren 
die niet in de grondwaarheid voorkwamen. Dit maakt de werkelijke precisie mogelijk hoger.

Hoewel de micro-gemiddelden een goed overzicht geven van de prestaties van het model, hangen de resultaten in de praktijk 
sterk af van de aard van de kritische objecten. 
Zo bleek het model het bijzonder moeilijk te hebben met het detecteren van kleine objecten (ampule, snoepje, spuit), zeker wanneer ze een laag contrast hebben met hun omgeving.
Dit leidde tot een lage recall voor deze objecten.
Ook werden er problemen vastgesteld bij het detecteren van objecten die meerdere componenten bevatten, zoals het infuus.
Wanneer men wil weten of een student naar een specifiek deel van een object heeft gekeken, zoals de infuuszak, 
zal het model dit object ook detecteren wanneer er gekeken wordt naar een ander deel, bijvoorbeeld de infuuspaal.
Dit kan, afhankelijk van de context, gezien worden als een vals-positief wat de precisie van het model verlaagt voor dat specifieke object.

De nauwkeurigheid van de duurmeting (2) is direct afhankelijk van de frame-per-frame correctheid van de objectidentificatie (1).
Waar objecten correct worden geïdentificeerd, kan de duur accuraat worden afgeleid door het aantal frames te tellen. 
Fouten in identificatie (vals-negatieven of -positieven) leiden echter direct tot onnauwkeurigheden in de duurmeting.

\section{Bijdrage en Meerwaarde}

De voornaamste bijdrage van deze bachelorproef ligt in de ontwikkeling van een werkend Proof-of-Concept.
Deze demonstreert de haalbaarheid van het geautomatiseerd analyseren van eyetrackingdata in een dynamische omgeving zoals het Zorglab.
Het was de bedoeling om een platform te creëren dat gemakkelijk uit te breiden en te onderhouden valt, zodat toekomstige onderzoekers 
en ontwikkelaars deze basis kunnen gebruiken voor verder onderzoek.
De methodologie voor het creëren van een grondwaarheidsdataset en de evaluatie van de analysepipeline 
levert een referentie voor toekomstige projecten binnen dit domein.

Hoewel het hier niet gaat om productieklare software die direct in het Zorglab kan worden ingezet, 
zet dit onderzoek wel een grote stap richting een geautomatiseerde evaluatiemethode.
In de toekomst zullen trainers en studenten kunnen profiteren van een meer doelgerichte en objectieve evaluatie van observatievaardigheden.
Het opleiden van elke student vereist een unieke aanpak, wat momenteel een hoge werkdruk met zich meebrengt voor docenten.
De geautomatiseerde evaluatie kan deze werkdruk verlichten door objectieve data te 
leveren die gericht zijn op de specifieke vaardigheden (of blinde vlekken) van elke student. 
Ook de maatschappij heeft voordeel bij een hogere zorgkwaliteit, door studenten beter voor te bereiden op de praktijk.
Deze bachelorproef legt de fundering voor een toekomst waarin technologie en zorgonderwijs hand in hand gaan,
en waar de hierboven vernoemde voordelen gerealiseerd kunnen worden.

\section{Reflectie en Discussie}

De behaalde resultaten zijn veelbelovend, met name de hoge precisie (circa 0.94, en in de werkelijkheid hoger) van het beste model. 
Een hoog aantal vals-positieven zou echter de bruikbaarheid van de software in het gedrang kunnen brengen, door 
een inaccuraat beeld te schetsen van de werkelijk bekeken objecten.
De lagere recall (circa 0.70) geeft aan dat niet alle bekeken objecten consistent werden gedetecteerd en dat er een marge is voor verbetering.
Dit was deels te verwachten, gezien de complexiteit van de taak (variërende objectgroottes, belichting, occlusie, snelle hoofdbewegingen).

Een onverwachte conclusie was dat er praktisch geen verschil was in de prestaties van de verschillende YOLOv11-objectdetectors,
ondanks de variërende datasetgroottes.
Dit gaf een indicatie dat de segmentatie van FastSAM de beperkende factor was,
en niet de objectdetectie op zich.
Om een beter beeld te krijgen van de impact van FastSAM, 
zou het interessant zijn om bij de vals-negatieven te onderzoeken of deze objecten toch gedetecteerd werden door de YOLOv11-objectdetector, 
maar niet door FastSAM.
Daarnaast heeft FastSAM twee parameters die in dit onderzoek niet werden geoptimaliseerd: 
\texttt{iou} en \texttt{conf}.
Een hogere \texttt{iou} en een lagere \texttt{conf} zouden kunnen leiden tot meer getrackte objecten, waardoor de recall eventueel kan worden verhoogd.
Langs de andere kant is er een kans op een lagere precisie, omdat er mogelijk meer fout-positieven worden gegenereerd.

De analyse van vals-positieven en -negatieven bracht interessante inzichten. 
Veel zogenaamde vals-positieven bleken toch correcte detecties te zijn die door inconsistenties 
tussen de FastSAM-output en de grondwaarheid niet als dusdanig werden geregistreerd. 
Dit wijst op een potentieel probleem in de methodologie voor het creëren van de grondwaarheid,
waarbij `bekeken' objecten werden geselecteerd op basis van de overlap tussen het blikpunt en de handmatig gelabelde segmentaties.

Een andere invalshoek is dat er potentieel een probleem is met de definitie van `bekeken' objecten.
Deze definitie van `overlapping', creëert echter een grijze zone.
Het is denkbaar dat een student visuele informatie van een object verwerkt, zelfs als het precieze blikpunt 
net buiten de grenzen van het gedetecteerde segment valt; bijvoorbeeld door perifeer zicht of een lichte 
onnauwkeurigheid in de eyetracker die de gedefinieerde marge overstijgt.
De huidige binaire aanpak (wel/niet bekeken) 
kan deze nuances moeilijk vatten en leidt mogelijk tot een onderschatting van daadwerkelijk geobserveerde objecten.
Toekomstig onderzoek zou kunnen overwegen om deze definitie te herzien,
bijvoorbeeld door te werken met de afstand tussen het blikpunt en de segmentatie in plaats van een binaire overlap.

\subsection{Toekomstig Onderzoek en Aanbevelingen}

Het is belangrijk te benadrukken dat deze resultaten geen volledig beeld geven hoe het 
systeem zal presteren onder alle mogelijke praktijkomstandigheden of met een ongelimiteerde variëteit aan objecten.
Het experiment dat uitgevoerd werd in het Zorglab was geen naturalistische setting,
maar een gecontroleerde omgeving met een beperkt aantal objecten.
Om de relevantie van het systeem verder te valideren, is het belangrijk om tests uit te voeren in een realistische setting,
waarbij studenten in een echte, relevante zorgsituatie worden geplaatst. 
Bovendien werd bij de evaluatie van de analysemethode en het trainen van de modellen uitsluitend gebruik gemaakt van de kalibratieopname 
waarbij de objecten zich tegen dezelfde achtergrond bevonden als tijdens de evaluatieopnames. 
Hoewel er ook een kalibratieopname met een afwijkende achtergrond werd gecreëerd, viel het analyseren van de impact hiervan buiten de scope van deze proef. 
Toekomstig onderzoek zou zich kunnen richten op het expliciet trainen en testen van de modellen met variërende achtergronden om de robuustheid 
van het systeem tegen contextuele veranderingen te beoordelen en te verbeteren.

Daarnaast is de ontwikkelde analysemethode, hoewel veelbelovend, momenteel redelijk complex. 
De combinatie van FastSAM en YOLOv11 is niet vanzelfsprekend en zorgt voor veel potentiële foutenbronnen.
Toekomstige modellen kunnen eventueel gebruik maken van een enkele, end-to-end benadering die zowel objectdetectie als segmentatie in één stap uitvoert.
Hoewel YOLOv11 ook tracking ondersteunt, is het nog niet mogelijk om segmentaties te verkrijgen binnen een trackingopdracht. 
Daardoor kan de overlap tussen het blikpunt en het object niet worden berekend.

Een andere interessante overweging is de beschikbaarheid van een grafische kaart (GPU) in het Zorglab.
Momenteel beschikt het Zorglab niet over een GPU, waardoor de ontwikkelde software niet kan worden gebruikt.
Bij het Zorglab worden niet enkel computervisie-modellen onderzocht, maar ook andere AI-toepassingen zoals spraakherkenning.
Ook deze toepassingen zouden kunnen profiteren van de rekenkracht van een sterke GPU.
De GPU die gebruikt werd in dit onderzoek, was een NVIDIA RTX 4090, wat snelle iteratie mogelijk maakte.
Met tragere GPU's zou het trainen van de modellen en het uitvoeren van de analyses veel langer duren. 
Aangezien AI-toepassingen steeds sterker worden en meer toepassingen bieden, 
is het aan te raden om in de toekomst te investeren in een krachtige GPU.

Op vlak van hardware is het ook belangrijk om stil te staan bij de resolutie van de camera van de eyetracker.
Bij de analyse werd vastgesteld dat het model moeite had met het detecteren van kleine objecten.
Het kan dus interessant zijn om de evolutie op de markt van de eyetrackers nauwlettend op te volgen.
Hierbij kan men kijken naar hogere resoluties of betere camera's die mogelijk meer details kunnen vastleggen.
Een andere optie voor het verbeteren van beeldkwaliteit is het toepassen van motion-deblurring technieken, 
die in eerdere onderzoeken al potentieel toonden \autocite{Cederin2023}.

Zoals eerder vermeld, is de ontwikkelde software modulair opgebouwd, zodat\\ nieuwe componenten gemakkelijk kunnen worden toegevoegd.
Studenten kunnen bij volgende bachelorproeven baat hebben bij de ontwikkelde labeling tool, om zo meer tijd te kunnen besteden aan het ontwikkelen van nieuwe analysecomponenten.
Indien deze componenten goed presteren, kunnen ze worden toegevoegd aan de bestaande software. 
Hierbij dient een analysemodule te worden ontwikkeld waarbij een trainer eyetrackingopnames kan selecteren en eventueel verschillende analysemethoden kan combineren.
Een voorbeeld hiervan is een analysemethode voor gezichtsherkenning, die kan bepalen of een student naar een specifieke persoon kijkt.
Tenslotte beschikt de software momenteel niet over een visualisatiemodule die de resultaten van de analyse(s) op een gebruiksvriendelijke manier presenteert.

