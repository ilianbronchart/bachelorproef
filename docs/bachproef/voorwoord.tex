%%=============================================================================
%% Voorwoord
%%=============================================================================

\chapter*{\IfLanguageName{dutch}{Woord vooraf}{Preface}}%
\label{ch:voorwoord}

Toen ik op het forum voor potentiële bachelorproefonderwerpen dit project rond computer visie in het Zorglab zag, was mijn interesse onmiddellijk gewekt. 
Na enkele jaren professioneel actief te zijn in de wereld van computer vision, zag ik hierin een unieke kans om 
mijn technische vaardigheden in te zetten voor een maatschappelijk relevant doel.
Bovendien bood het een gelegenheid om meer te leren over de nuances en valkuilen binnen de zorgverlening – een wereld die tot dan toe relatief onbekend voor me was.

Het zaadje waaruit dit project is gegroeid, werd geplant door dhr. Jorrit Campens, lector en onderzoeker aan HOGENT en een drijvende kracht achter het Zorglab.
Zijn domeinexpertise in de zorg vormde een perfecte aanvulling op mijn technische achtergrond. 
Onze samenwerking voelde als een natuurlijke synergie, waarbij hij me doorheen het hele traject, van gerichte feedback voorzag.
Ik herinner me nog goed ons eerste gesprek in het Zorglab, waarbij dhr. Campens zijn droom schetste van een geautomatiseerde analysemethode voor observaties in zorgsimulaties.
Hij omschreef het toen als `naar de maan gaan', een uitdaging die me meteen aansprak. Voor zijn enthousiasme en stimulans, ben ik hem dan ook bijzonder dankbaar.

Toegegeven, het project was ambitieus en uitdagend, zeker in combinatie met mijn deeltijdse werk als softwareontwikkelaar gedurende het semester.
Desondanks zag ik het als een perfecte kans om de diverse vaardigheden die ik doorheen de jaren heb opgebouwd, samen te brengen.
Mijn IT-reis begon met het ontwikkelen van games in het middelbaar en evolueerde naar het bouwen van webapplicaties en het verkennen van frontend-tecnologieën.
Binnen mijn opleiding aan HOGENT koos ik bewust voor de minor Data \& AI, een domein waarin ik mijn kennis nog verder in wilde verdiepen.
Deze bachelorproef voelde dan ook als een culminatiepunt, waar ik mijn passie voor frontend, backend en data science kon verenigen.

Mijn dank gaat ook uit naar mijn promotor, dhr. Bert Van Vreckem, voor zijn waardevolle inhoudelijke feedback en zijn beschikbaarheid voor overlegmomenten gedurende het traject.
Een speciaal woord van dank wil ik richten aan mijn bonusvader, Dirk Coussement.
Als auteur voor een lokaal blad heeft hij een scherp oog voor taal en heeft hij mij enorm geholpen bij het nalezen van deze bachelorproef.

Tenslotte wil ik ook de studenten bedanken die de tijd namen om deel te nemen aan het experiment in het Zorglab. 
Zonder hun medewerking was het verzamelen van de nodige data niet mogelijk geweest.

Ik hoop dat deze bachelorproef een bruikbare eerste stap vormt naar een meer objectieve en efficiënte manier om observatievaardigheden in zorgopleidingen te evalueren.