%%=============================================================================
%% Samenvatting
%%=============================================================================

% TODO: De "abstract" of samenvatting is een kernachtige (~ 1 blz. voor een
% thesis) synthese van het document.
%
% Een goede abstract biedt een kernachtig antwoord op volgende vragen:
%
% 1. Waarover gaat de bachelorproef?
% 2. Waarom heb je er over geschreven?
% 3. Hoe heb je het onderzoek uitgevoerd?
% 4. Wat waren de resultaten? Wat blijkt uit je onderzoek?
% 5. Wat betekenen je resultaten? Wat is de relevantie voor het werkveld?
%
% Daarom bestaat een abstract uit volgende componenten:
%
% - inleiding + kaderen thema
% - probleemstelling
% - (centrale) onderzoeksvraag
% - onderzoeksdoelstelling
% - methodologie
% - resultaten (beperk tot de belangrijkste, relevant voor de onderzoeksvraag)
% - conclusies, aanbevelingen, beperkingen
%
% LET OP! Een samenvatting is GEEN voorwoord!

%%---------- Nederlandse samenvatting -----------------------------------------
%
% TODO: Als je je bachelorproef in het Engels schrijft, moet je eerst een
% Nederlandse samenvatting invoegen. Haal daarvoor onderstaande code uit
% commentaar.
% Wie zijn bachelorproef in het Nederlands schrijft, kan dit negeren, de inhoud
% wordt niet in het document ingevoegd.

\IfLanguageName{english}{%
\selectlanguage{dutch}
\chapter*{Samenvatting}


\selectlanguage{english}
}{}

%%---------- Samenvatting -----------------------------------------------------
% De samenvatting in de hoofdtaal van het document

\chapter*{\IfLanguageName{dutch}{Samenvatting}{Abstract}}

Deze bachelorproef richt zich op het verbeteren van de observatievaardigheden van studenten in de zorg, specifiek binnen het 360° Zorglab van HOGENT. 
Observatievaardigheden zijn belangrijk voor zorgverleners, niet alleen voor het stellen van nauwkeurige diagnoses, maar ook voor de juiste ondersteuning en communicatie met patiënten. 
Momenteel ervaren studenten vaak moeilijkheden met het herkennen van kritische objecten tijdens zorgsimulaties, zoals het opmerken van een colafles bij een diabetespatiënt.
Eerdere evaluatiemethodes steunden op zelfrapportage en observatie door docenten, maar deze zijn subjectief en onbetrouwbaar. Hierdoor ontstaat de noodzaak voor een objectieve en efficiënte methode.
\par
Deze bachelorproef onderzoekt hoe eyetracking-technologie van Tobii Glasses gecombineerd kan worden met state-of-the-art Machine Learning modellen uit computervisie om observatievaardigheden objectief te analyseren en visualiseren. 
De centrale onderzoeksvraag luidt: `Hoe kunnen computervisie-modellen geïntegreerd worden met eyetrackingdata van Tobii Glasses om observatieprestaties van studenten in het 360° Zorglab automatisch te analyseren en te visualiseren?'.
\par
Om dit te realiseren, wordt een proof-of-concept (PoC) softwareoplossing ontwikkeld volgens een agile methodiek met iteratieve cycli.
Als ondersteuning voor de PoC wordt een literatuurstudie uitgevoerd om na te gaan welke objectdetectie- en segmentatiemodellen geschikt zijn voor dit project.
Ook wordt er gekeken naar bestaande implementaties van computervisie binnen eyetracking applicaties en hoe deze geïntegreerd kunnen worden.
De PoC omvat het verzamelen van eyetrackingdata, het definiëren en labelen van kritische objecten, en de implementatie van een data-architectuur waarin modellen zoals YOLO, DINOv2, SAM en FastSAM worden toegepast. 
Daarnaast wordt onderzocht hoe effectief en nauwkeurig deze modellen en de resulterende metrieken zijn in het identificeren van waargenomen objecten en hun kijkduur.
\par
Op basis van het onderzoek wordt verwacht dat de ontwikkelde aanpak een aanzienlijke verbetering zal betekenen in de betrouwbaarheid en efficiëntie van het evaluatieproces.
