%%=============================================================================
%% Methodologie
%%=============================================================================

\chapter{\IfLanguageName{dutch}{Methodologie}{Methodology}}%
\label{ch:methodologie}

%% TODO: In dit hoofstuk geef je een korte toelichting over hoe je te werk bent
%% gegaan. Verdeel je onderzoek in grote fasen, en licht in elke fase toe wat
%% de doelstelling was, welke deliverables daar uit gekomen zijn, en welke
%% onderzoeksmethoden je daarbij toegepast hebt. Verantwoord waarom je
%% op deze manier te werk gegaan bent.
%% 
%% Voorbeelden van zulke fasen zijn: literatuurstudie, opstellen van een
%% requirements-analyse, opstellen long-list (bij vergelijkende studie),
%% selectie van geschikte tools (bij vergelijkende studie, "short-list"),
%% opzetten testopstelling/PoC, uitvoeren testen en verzamelen
%% van resultaten, analyse van resultaten, ...
%%
%% !!!!! LET OP !!!!!
%%
%% Het is uitdrukkelijk NIET de bedoeling dat je het grootste deel van de corpus
%% van je bachelorproef in dit hoofstuk verwerkt! Dit hoofdstuk is eerder een
%% kort overzicht van je plan van aanpak.
%%
%% Maak voor elke fase (behalve het literatuuronderzoek) een NIEUW HOOFDSTUK aan
%% en geef het een gepaste titel.

De bachelorproef zal worden uitgevoerd volgens een agile aanpak, waarbij iteratieve cycli (sprints) en overlappende fasen worden gebruikt om flexibiliteit en continue verbetering mogelijk te maken. 
Deze aanpak zorgt ervoor dat verschillende onderdelen van het project parallel kunnen verlopen en snel kunnen worden aangepast op basis van tussentijdse bevindingen.
Nieuwe taken worden doorheen de sprints toegevoegd aan een backlog binnen een Trello-bord, en worden doorheen de sprints opgepakt en afgerond.
Het doel voor de PoC is om een user-flow te ontwikkelen die het mogelijk maakt om de volgende zaken uit te voeren:
\begin{itemize}
    \item 1. Het ophalen van eyetrackingdata van de Tobii Glasses
    \item 2. Het definiëren / labelen van kritische objecten die geobserveerd moeten worden door de studenten.
    \item 3. Het analyseren van van de data met objectdetectie-, segmentatie-, en image-embedding-modellen. 
    Deze stap wordt hierna de data-architectuur genoemd.
    \item 4. Het visualiseren van de resultaten van de analyse via een metriek die de blik-punten van de studenten koppelt aan de gedetecteerde objecten.
\end{itemize}
\par
De voorkeur gaat naar het gebruik van voorgetrainde modellen om de nood aan hertrainen te minimaliseren en zo de gebruikerservaring te verbeteren. 
Toch kan het zijn dat voorgetrainde modellen niet voldoen aan de specifieke noden van het project. 
Daarom kan indien nodig de PoC uitgebreid worden met extra functionaliteiten zoals het finetunen van modellen met simulatiespecifieke data.

\section{Tijdsplanning}

De bachelorproef is gepland van 10 februari 2025 tot en met 23 mei 2025, met een totaal van 7 sprints van 2 weken.
De activiteiten binnen elke sprint omvatten, in geen specifieke volgorde:
\begin{itemize}
    \item Literatuurstudie
    \item Dataverzameling en labeling
    \item Modelselectie en implementatie
    \item Schrijven van Python Notebooks voor experimenten
    \item Ontwikkelen van de data-architectuur binnen de PoC
    \item Ontwikkelen van de user interface voor de PoC
    \item Uitbreiding van de PoC met nieuwe functionaliteiten
    \item Unit-testen schrijven voor de PoC
    \item Meetings met de co-promotor voor feedback en verfijning van de oplossing
    \item Documentatie van het proces en resultaten
    \item Uitschrijven, aanpassen van het proefschrift
\end{itemize}
Toch zal de periode opgesplitst worden in de volgende sprints:

\subsection{Sprint 1}
\begin{itemize}
    \item Relevante modellen zijn verzameld en uitgetest binnen een Python Notebook.
    \item Er is huisgemaakte data verzameld via geleende Tobii Glasses uit het Zorglab.
    \item Potentiële segmentatie- en detectiepipeline architecturen voor de PoC zijn uitgestippeld.
    \item Er is een user interface binnen de PoC om opnames op te halen van de Tobii Glasses.
\end{itemize}
\subsection{Sprint 2}
\begin{itemize}
    \item Het literatuurstudie onderdeel van het proefschrift is uitgewerkt.
    \item De abstract en inleiding van het proefschrift zijn geschreven.
\end{itemize}
\subsection{Sprint 3}
\begin{itemize}
    \item Er is een demo gegeven aan de co-promotor voor een kandidaat architectuur van de PoC.
    \item De feedback van de co-promotor is verwerkt en gedocumenteerd.
    \item Er is een metriek ontwikkeld voor het meten van waar een student naar heeft gekeken en voor hoe lang.
    \item Er zijn verschillende data-architecturen geëvalueerd.
\end{itemize}
\subsection{Sprint 4}
\begin{itemize}
    \item Er bestaat een manier om belangrijke objecten te labelen binnen de PoC
    \item De segmentatie- en detectiepipeline is geïntegreerd binnen de user interface van de PoC
    \item Er is een onderzoeksmethodologie opgesteld voor het evalueren van de metrieken die berekend worden door de PoC.
\end{itemize}
\subsection{Sprint 5}
\begin{itemize}
    \item Er is data verzameld binnen het Zorglab van HOGENT volgens de opgestelde onderzoeksmethodologie.
    \item De opgehaalde data is gelabeld.
\end{itemize}
\subsection{Sprint 6}
\begin{itemize}
    \item De vergelijking tussen verschillende data-architecturen is toegevoegd aan het proefschrift op basis van de data uit sprint 5.
    \item De metrieken die berekend worden door de PoC zijn geëvalueerd.
\end{itemize}
\subsection{Sprint 7}
\begin{itemize}
    \item Er bestaat een visualisatieinterface binnen de PoC die de metrieken toont voor geselecteerde eyetracking opnames.
    \item De conclusie en aanbevelingen van het proefschrift zijn geschreven.
    \item Het proefschrift is nagelezen en aangepast.
\end{itemize}    

\section{Tools en Technologieën}

\begin{itemize} 
  \item \textbf{Programmeertaal en Frameworks}: Python, PyTorch, TensorFlow voor machine-learning modellering en implementatie. 
  \item \textbf{Programmeerstack van de PoC}: FastAPI voor de backend, HTMX en Jinja2 voor de frontend.
  \item \textbf{Data Verwerking en Visualisatie}: OpenCV voor videoverwerking, Matplotlib voor datavisualisatie. 
  \item \textbf{Ontwikkelomgeving}: Visual Studio Code, Python Notebooks voor experimenten, Git voor versiebeheer. Poetry voor dependency management.
  \item \textbf{GPU}: At-home Nvidia RTX 4090 en eventuele GPUs van HoGent.
  \item \textbf{Eyetracking}: Tobii Eyetracking Glasses, en Tobii Pro Glasses 3 Controller App.
  \item \textbf{Planning}: Trello voor projectmanagement.
\end{itemize}
